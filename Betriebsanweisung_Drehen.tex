\documentclass[fontsize=9pt]{scrartcl}
\pagestyle{empty}
\usepackage{lmodern}
\renewcommand*\familydefault{\sfdefault}
\usepackage[utf8]{inputenc}
\usepackage[T1]{fontenc}
\usepackage[ngerman]{babel}
\usepackage[babel,german=quotes]{csquotes}
\usepackage{graphicx}
\usepackage{xcolor}
\usepackage{geometry}
\geometry{a4paper, top=15mm, left=20mm, right=20mm, bottom=15mm, headsep=0mm, footskip=0mm}

\usepackage{tikz}
\makeatletter
\setkomafont{section}{\color{white}
    \bfseries\Large
    \begin{tikzpicture}[overlay]
        \draw[blue,fill=blue] (-1.5cm,-6pt) rectangle (\paperwidth, 13pt)
        (\linewidth,16.4pt);
	\end{tikzpicture}}

\usepackage{background}
\backgroundsetup{
	scale=1,	%% Größe (so lassen)
	angle=0,	%% Ausrichtung (Winkel)
	opacity=1,  %% Deckkraft
	contents={\includegraphics[width=\paperwidth,height=\paperheight]{img/blauer_rand.pdf}}
}

\newenvironment{smallitemize}{\begin{itemize}\itemsep -3pt}{\end{itemize}}



\begin{document}

% blauer Rand

\begin{center}
	\LARGE{Betriebsanweisung \enquote{Drehbank} Wabeco CC-D6000hs}
\end{center}


%\section{\bgvbox{Anwendungsbereich}}
%\section{Anwendungsbereich}
\begin{center}
	Diese Betriebsanweisung fasst die wichtigsten Gefahren und Regeln zusammen.\\
	Für die Bedienung der Drehbank ist eine unterschriebene Einweisung nötig.\\
	Weitere Informationen finden sich im ausführlichen Einweisungstext.\\
\end{center}

\section{Gefahren für Mensch und Umwelt}

\begin{smallitemize}
	\item Getroffenwerden vom wegfliegendem Werkstück, von wegfliegenden Teilen, Spänen usw.
	\item Sich schneiden, stechen usw. an Werkzeug, Werkstück, Spänen.
	\item Intensiver Hautkontakt mit Kühlschmierstoff kann zu Hautschäden führen.
	\item Schon geringfügige Hautverletzungen, z.B. durch Metallteilchen, erhöhen das Risiko einer kühlschmierstoff-bedingten Hauterkrankung.
\end{smallitemize}

\section{Schutzmaßnahmen und Verhaltensregeln}

\begin{itemize}
	\item Generell:
	\begin{smallitemize}
		\item Vor Arbeitsbeginn Wartungsplan beachten, Maschine auf Mängel und mögliche Probleme kontrollieren.
		\item Sicherstellen, dass sich nur der Bediener und höchstens eine eingewiesene Hilfsperson in der Sicherheitszone befinden, Sicherheitszone absperren.
		\item Zum Werkzeugwechsel, Messen, Reinigen usw. Stillstand aller Maschinenteile sicherstellen. (Handbetrieb: Drehrichtung auf 0 stellen)
		\item Maschine nach Gebrauch abschalten und Hauptschalter auf Stellung „0“ stellen.
		\item Späne nur bei stehender Maschine mit geeignetem Werkzeug (z.B. Pinsel, Staubsauger; nicht: Hand, Druckluft) entfernen
		
	\end{smallitemize}
	\item Im Betrieb:
	\begin{smallitemize}
		\item Werkstück fest im Futter spannen und Spannschlüssel abziehen.
		\item Stangenmaterial darf nicht aus der Maschine ragen.
		\item Lose Teile (Spannschlüssel, Werkstücke, etc.) nicht im Gefahrenbereich beweglicher Maschinenteile lagern.
		\item Handschuhe dürfen beim Drehen nicht getragen werden.
		
	\end{smallitemize}
	\item Im Handbetrieb:
	\begin{smallitemize}
		\item Lange Haare durch Mütze, Haarnetz o.Ä. verdecken. Zopfgummi u.Ä. sind nicht ausreichend!
		\item Eng anliegende, geschlossene Arbeitskleidung tragen, ggf. Ärmel nach innen aufrollen.
		\item Lose Teile, Uhren,  Ringe, Arm- und Halsschmuck, Krawatten, Schals usw. ablegen.
		\item Grundsätzlich Gesichtsschild oder Schutzbrille tragen. Eine normale Brille ist keine Schutzbrille!
		\item Benachbarte Arbeitsplätze nicht durch spritzenden Kühlschmierstoff, wegfliegende Späne usw. gefährden.
		\item Ausreichend Abstand von allen drehenden Teilen halten. Handbearbeitung mit Schleifpapier ist nur mit dem speziellen Halter zulässig. Sonstiges Handwerkzeug wie eine Feile darf nicht verwendet werden.
		\end{smallitemize}
\end{itemize}

\section{Verhalten bei Störungen und im Gefahrenfall}
\begin{smallitemize}
	\item Bei Schäden oder Störungen an der Maschine: Ausschalten und Betreuer informieren. Schadensmeldung sichtbar an der Maschine anbringen.
	\item Rutschgefahr (z.B. durch Kühlschmiermittel, Späne) beseitigen.
	\item Schäden nur vom Fachmann beseitigen lassen.
\end{smallitemize}

\section{Verhalten bei Unfällen - Erste Hilfe}
\begin{smallitemize}
	\item Maschine abschalten. NOT-AUS drücken
	\item Betreuer informieren. Gegebenenfalls Rettungsdienst rufen.
	\item Verletzten betreuen.
\end{smallitemize}

\section{Instandhaltung, Entsorgung}
\begin{smallitemize}
	\item Nach Abschluss der Arbeiten Späne sortenrein in die Sammelbehälter entsorgen. Gemischte Späne sind Restmüll.
	\item Maschine bei Arbeitsende reinigen und Wartungsplan beachten.
		\vspace{5mm}
	\item Für die Instandhaltung ist zuständig:...............................
\end{smallitemize}


\end{document} 